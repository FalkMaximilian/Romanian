\documentclass[11pt, oneside]{article}
\usepackage{geometry}
\geometry{letterpaper}
\usepackage{graphicx}		
\usepackage[normalem]{ulem}
\usepackage{setspace}
\usepackage{amssymb}
\usepackage[dvipsnames]{xcolor}
\usepackage{makeidx}
\usepackage{ragged2e}
\usepackage{amsmath}
\usepackage{hyperref}
\usepackage{array}
\renewcommand{\familydefault}{\sfdefault}
\usepackage[romanian]{babel}

\title{Română A1.1}
\author{Maximilian Falk}
\date{\today}
\makeindex
%
\newcolumntype{R}[1]{>{\RaggedLeft\arraybackslash}p{#1}}
\newcolumntype{C}[1]{>{\centering\let\newline\\\arraybackslash\hspace{0pt}}m{#1}}
%
% Ändert die Höhe der Tabelle
%
\begin{document}
\maketitle
\tableofcontents
%
%
\newpage
\section{Lecție}
%
\subsection{Pronunție - Betonung}
Im Rumänischen werden kombinierte Vokale "schwingend" ausgesprochen.\\
\newline
Das "R" wird im Rumänischen generell rollend ausgeprochen. (Zunge vorne am Gaumen)\\
\newline
ă, Ă $\rightarrow$ Bös\underline{e}\\
z, Z $\rightarrow$ \underline{S}ie\\
v, V $\rightarrow$ \underline{V}ase\\
â, Â $\rightarrow$ Dieser Laut existiert im Deutschen nicht. Dunkler als ă, Ă.\\
ș, Ș $\rightarrow$ Sch\\
ț, Ț $\rightarrow$ Z\\
ci $\rightarrow$ Tschi\\
c $\rightarrow$ k\\
j $\rightarrow$ In\underline{ge}nieur\\
gi $\rightarrow$ \underline{J}eans\\
\newline
î, Î ist der selbe Laut wie â, Â. Das î wird am Wortanfang und Ende verwednet, 
das â findet man hauptsächlich im Wort. Aussprache: \url{https://youtu.be/Y4p9M-IjCVE?t=206}
\subsection{A fi - Zu sein}
\index{a fi}
\begin{center}
    \begin{tabular}{ |p{2cm}|p{2cm}||l|l|  }
        \hline
        \multicolumn{4}{|c|}{a fi - zu sein} \\
        \hline
        \hline
        Eu sunt & Ich bin & Noi suntem & Wir sind\\
        \hline
        Tu ești & Du bist & Voi sunteți & Ihr seid\\
        & & Dumneavoastră sunteți & Sie sind (Höflichkeitsform)\\
        \hline
        El este/e & Er ist & Ei sunt & Sie sind (Sobald 1 Mann in der Gruppe)\\ 
        Ea este/e & Sie ist & Ele sunt & Sie sind (Gruppe nur aus Frauen)\\
        \hline
       \end{tabular}
\end{center}
%
\subsection{Frază și Vocabulă}
\begin{center}
  \begin{tabular}{ | l | l | } 
    \hline
    Bună ziua! & Guten Tag! (Formal)\\ 
    \hline
    Bună & Hallo (Informal)\\ 
    \hline
    Salut & Hallo (Informal)\\ 
    \hline
    Bună seara! & Guten Abend!\\
    \hline
    Bună dimineața! & Guten Morgen!\\
    \hline
    Noapte bună! & Gute Nacht!\\
    \hline
    Pa! & Tschüss! (Informal)\\
    \hline
    La revedere! & Auf Wiedersehen! (Formal)\\
    \hline
    Pe mâine! & Bis morgen!\\
    \hline
    O zi bună! & Einen schönen Tag!\\
    \hline
    O seară bună! & Einen schönen Abend!\\
    \hline
    O săptămână bună! & Eine schöne Woche!\\
    \hline
    Week-end plăcut! & Ein schönes Wochenende!\\
    \hline
    Te rog! & Ich bitte dich!\\
    \hline
    Vă rog! & Ich bitte Sie!\\
    \hline
    Mult succes! & Viel Erfolg!\\
    \hline
    Bine ați venit! & Willkommen! (Wörtich: Gut dass Sie gekommen sind)\\
    \hline
    Bine ai venit! & Willkommen! (Du; Informal)\\
    \hline
    Poftim! & Nimm das!\\
    \hline
    Ce mai faci? & Wie geht es dir?\\
    \hline
    Ce mai faceți? & Wie geht es Ihnen?\\
    \hline
    Îmi pare bine! & Ich freue mich!\\
    \hline
    La fel & Gleichfalls\\
    \hline
    Așa și așa & Es geht so\\
    \hline 
    Ce ești tu de profesie? & Was machst du beruflich?\\
    \hline
    Cu plăcere & Gerne; Kein Ding\\
    \hline
    Mulțumesc & Danke\\
    \hline
    Da & Ja\\
    \hline
    Nu & Nein\\
    \hline
    Cum te cheamă & Wie heißt du?\\
    \hline
    Mă cheamă/numesc \dots & Ich heiße \dots\\
    \hline
    Niște & Einige\\
    \hline
  \end{tabular}
\end{center}
\pagebreak
%
%
\section{Lecție}

\subsection{Numere - Die Zahlen}
1 - unu, 2 - doi, 3 - trei, 4 - patru, 5 - cinci, 6 - șase, 7 - șapte, 8 - opt,
9 - nouă, 10 - zece\\
\newline
11 - unsprezece, 12 - doisprezece, 13 - treisprezece, 14 - paisprezece, 15 - 
cincisprezece, 16 - șaisprezece, 17 - șaptesprezece, 18 - optsprezece, 19 - 
nouăsprezece, 20 - douăzeci\\
\newline
21 - douăzeci și unu, 22 - douăzeci și doi/două, \dots\\
\newline
30 - treizeci, 40 - patruzeci, 50 - cincizeci, 60 - șaizeci, \dots, 100 - o sută, 200 - două sute\\
\newline
1000 - o mie, 2000 - Două mii, \dots
%
\subsection{Zile lucrătoare - Wochentage}
\begin{itemize}
  \item Luni - Montag
  \item Marți - Dienstag
  \item Miercuri - Mittwoch
  \item Joi - Donnerstag
  \item Vineri - Freitag
  \item Sâmbăta - Samstag
  \item Duminică - Sonntag
\end{itemize}
%
\subsection{Cuvinte Interogative - Fragewörter}
\begin{itemize}
  \item Ce $\rightarrow$ Was\\
  Ce zi este azi? $\rightarrow$ Welcher Tag ist heute?
  \item De ce $\rightarrow$ Warum\\
  De ce pleci $\rightarrow$ Warum gehst du?
  \item Cine $\rightarrow$ Wer\\
  Pentru cine? $\rightarrow$ Für wen?
  \item Cum $\rightarrow$ Wie, Wieso\\
  Cum te cheamă $\rightarrow$ Wie heißt du?
  \item Când $\rightarrow$ Wann\\
  Când ești acasă? $\rightarrow$ Wann bist du zu Hause?
  \item Cât $\rightarrow$ Wie viel\\
  Cât este/e ceasul? $\rightarrow$ Wie viel Uhr ist es?\\
  Cât costa 10 beri $\rightarrow$ Wie viel kosten 10 Bier?
  \item Unde $\rightarrow$ Wo, Wohin\\
  Unde sunt colegii? $\rightarrow$ Wo sind die Kollegen?
  \item De unde $\rightarrow$ Von wo\\
  Du unde sunteți? $\rightarrow$ Von wo sind Sie?
\end{itemize}
%
\subsection{Plural}
t + i am Ende eines Wortes $\rightarrow$ ți\\
s + t + i am Ende eines Wortes $\rightarrow$ ști\\
s + i am Ende eines Wortes $\rightarrow$ și\\
\begin{center}
  \begin{tabular}{ |l|l||l|l|  }
      \hline
      Un profesor & Doi profesor\textcolor{Red}{i} & O profesoară & Două profesoar\textcolor{Red}{e}\\
      \hline
      Un avocat & Doi avoca\textcolor{Red}{ți} & O avocată & Două avocat\textcolor{Red}{e}\\
      \hline
      Un pianist & Doi piani\textcolor{Red}{ști} & &\\
      \hline
      Un pas & Doi pa\textcolor{Red}{și} & &\\
      \hline
     \end{tabular}
\end{center}
%
Der unbestimmte Artikel im Plural lautet: \emph{Niște}\\
\newline
\emph{Aici sunt niște studenți și niște studente.} $\rightarrow$
Hier sind einige Studenten und einige Studentinnen.
%
\subsection{A avea - Zu haben}
\index{a avea}
\begin{center}
  \begin{tabular}{ |p{2cm}|p{2cm}||l|l|  }
      \hline
      \multicolumn{4}{|c|}{a avea - zu haben} \\
      \hline
      \hline
      Eu am & Ich habe & Noi avem & Wir haben\\
      \hline
      Tu ai & Du hast & Voi aveți & Ihr habt\\
      & & Dumneavoastră aveți & Sie haben (Höflichkeitsform)\\
      \hline
      El are & Er hat & Ei au & Sie haben (Sobald 1 Mann in der Gruppe)\\ 
      Ea are & Sie hat & Ele au & Sie haben (Gruppe nur aus Frauen)\\
      \hline
     \end{tabular}
\end{center}
%
\subsection{Frază și Vocabulă}
\begin{center}
  \begin{tabular}{ | p{6cm}| p{6cm} | } 
    \hline
    Ce (zi) e azi? & Welcher Tag ist heute?\\
    \hline
    Azi/Astăzi &  Heute\\
    \hline
    Pe curând! & Bis bald!\\
    \hline
    Mâine & Morgen\\
    \hline
    Poimâine & Übermorgen\\
    \hline
    Cine & Wer \\
    \hline
    Ce & Was \\
    \hline
    Cum & Wie \\
    \hline
    Unde & Wo/Wohin\\
    \hline
    Ce păcat! & Wie schade!\\
    \hline
    Îmi pare rău & Es tut mir leid\\
    \hline
    Și ție! & Dir auch!\\
    \hline
    A avea nevoie & Zu brauchen\\
    \hline
    Ședința & Sitzung \\
    \hline
    Oraș & Stadt \\
    \hline
    Calculator & Computer \\
    \hline
    Cât e ceasul & Wie spät ist es?\\
    \hline
    Micul dejun & Frühstück\\
    \hline
    Prânz & Mittagessen\\
    \hline
    Cina & Abendessen\\
    \hline
    Dacă & Wenn\\
    \hline
  \end{tabular}
\end{center}
%
\pagebreak
\section{Lecție}
%
\subsection{Oră - Uhrzeit}
Grundsätzlich kann man immer \emph{și} verwenden um eine Uhrzeit anzugeben.
Ein Viertel ist \emph{un sfert}. \emph{Un sfert} kann verwendet werden um viertel vor 
und viertel nach zu sagen. Für halb verwendet man \emph{jumatătate} oder \emph{jumate}.\\
\newline
Von 1-19 wird ohne \emph{de} gezählt $\rightarrow$ Am 17 scaune.\\
Ab 20 wird mit \emph{de} gezählt $\rightarrow$ Am 20 de scaune.\\
\newline
%
12:05 $\rightarrow$ douăsprezece și cinci (minute)\\
12:17 $\rightarrow$ douăsprezece și șaptesprezece (minute)\\
12:25 $\rightarrow$ douăsprezece și douăzeci și cinci (de minute)\\
\newline
12:15 $\rightarrow$ douăsprezece și cincisprezece / douăsprezece și un sfert\\
12:30 $\rightarrow$ douăsprezece și treizeci / douăsprezece și jumatătate\\
\newline
12:35 $\rightarrow$ douăsprezece și treizeci și cinci / unu fără douăzeci și cinci (de minute)\\
12:45 $\rightarrow$ douăsprezece și patruzeci și cinci / unu fără un sfert\\
%
%
\subsection{Verbi}
\begin{center}
  
  Es gibt im Rumänischen 4 Gruppen für regelmäßige Verben.\\
  \emph{a urc\textcolor{Red}{a}, a cobor\textcolor{Red}{î}, a ved\textcolor{Red}{ea}, a merg\textcolor{Red}{e}, a fug\textcolor{Red}{i}}\\
  \begin{tabular}{ |C{1.4cm}|C{1.4cm}||C{1cm}|C{1cm}| C{1cm}| }
    \hline
    -a & -î & -ea & -e & -i \\
    \hline
    \hline
    \multicolumn{5}{|c|}{Endung fällt weg}\\
    \hline
    \multicolumn{5}{|c|}{-i}\\
    \hline
    \multicolumn{2}{|c||}{-ă} & \multicolumn{3}{c|}{-e}\\
    \hline
    -ăm & -âm & -em & -em & -im\\
    \hline
    -ați & -âți & -eți & -eți & -iți\\
    \hline
    \multicolumn{2}{|c||}{-ă} & \multicolumn{3}{c|}{Endung fällt weg}\\
    \hline
  \end{tabular}
\end{center}
\newpage
\begin{center}
Dazu gibt es auch auch 3 Untergruppen.\\
\emph{a lucra-\textcolor{Red}{ez}, a citi-\textcolor{Red}{esc}, a hotărî-\textcolor{Red}{ăsc}}\\
  \begin{tabular}{ |C{1.4cm}|C{1.4cm}|C{1.4cm}| }
    \hline
    -a & -i & -î\\
    \hline
    \hline
    -ez & -esc & -ăsc\\
    \hline
    -ezi & -ești & -ăști\\
    \hline
    -ează & -ește & -ăște\\
    \hline
    -ăm & -im & -îm\\
    \hline
    -ăți & -iți & îți\\
    \hline
    -ează & -esc & -ăsc\\
    \hline
  \end{tabular}\\
\end{center}
%
\subsection{Frăza și Vocabulă}
\begin{center}
  \begin{tabular}{ | p{6cm}| p{6cm} | } 
    \hline
    O ora - Două ore & Stunde\\
    \hline
    Un ceas - Două ceasuri & Uhr\\
    \hline
    O pagină - Două pagini & Seite\\
    \hline
    La ce pagină? & Auf welcher Seite?\\
    \hline
    de la \dots până la \dots & Von (Uhrzeit) bis (Uhrzeit)\\
    \hline
    Un taxi - Două taxiuri & Taxi\\
    \hline
    Un mail - Două mailuri & Mail\\
    \hline
    O activitate - Două activități & Tätigkeit\\
    \hline
    O dimineață - Două dimineți & Vormittag, Morgens\\
    \hline
    Un oraș - Două orașe & Stadt\\
    \hline
    Pe drum & Unterwegs\\
    \hline
    Ești pe drum? & Bist du unterwegs?\\
    \hline
    Un drum - Două drumuri & Weg\\
    \hline
    Pe cuvând & Auf mein Wort! Ich schöre es!\\
    \hline
    Un calculator - Două calculatoare & Computer\\
    \hline
    Dacă & Wenn, ob\\
    \hline
    O mașină - Două mașini & Auto\\
    \hline
    Un film - Două filme & Film\\
    \hline
    O familie - Două familii & Familie\\
    \hline
    Fiecare & Jede/r/s\\
    \hline
    O propoziție - Două propoziții & Satz\\
    \hline
    Hai! & Komm! Lass uns machen!\\
    \hline
    O țuică & (Pflaumen)Schnapps\\
    \hline
    O vișinată & Sauerkirschlikör\\
    \hline
    O bere - Două beri & Bier\\
    \hline
    Un vin - Două vinuri & Wein\\
    \hline
    O apă & Wasser\\
    \hline
  \end{tabular}
\end{center}

\begin{center}
  \begin{tabular}{ | p{6cm}| p{6cm} | } 
    \hline
    Mi-e sete & Ich hab Durst\\
    \hline
    Mi-e foame & Ich hab Hunger\\
    \hline
    Pe jos & Zu Fuß\\
    \hline
    A avea treabă & Beschäftigt sein\\
    \hline
    A face cumpărături & Einkaufen\\
    \hline
    O întâlnire - Două întâlniri & Verabredung\\
    \hline
    Acolo & Da, Dort\\
    \hline
    O masă - Două mese & Tisch\\
    \hline
    Un telefon - Două telefoane & Telefon\\
    \hline
    Un caiet - Două caiete & Heft\\
    \hline
    Un creion - Două creioane & Bleistift\\
    \hline
    Un pix - Două pixuri & Kuli\\
    \hline
    Un televizor - Două televizoare & Fernseher\\
    \hline
    Un bilet - Două bilete & Ticket\\
    \hline
    Un ziar - Două ziare & Zeitung\\
    \hline
    O foaie - Două foi & Blatt(Papier)\\
    \hline
    Împreună & Zusammen, miteinander\\
    \hline
    Când pleci de acasă? & Wann gehst du von zu Hause weg?\\
    \hline
    Iar tu? & Und du?\\
    \hline
    Dar & Aber\\
    \hline
    Mai (adv.) & Mehr\\
    \hline
    Nou & Neu\\
    \hline
    Pentru & Für\\
    \hline
    un client - doi clienți & Kunde\\
    o clientă - două cliente & Kundin\\
    \hline

    \hline
  \end{tabular}
\end{center}
%
\pagebreak
\section{Lecție}
\subsection{Lunile și Dată - Monate und Datum}
\begin{center}
  \begin{tabular}{ | C{1cm}| p{3cm} || C{1cm} | p{3cm} |} 
    \hline
    1. & Ianuarie & 7. & Iulie\\
    \hline
    2. & Februarie & 8. & August\\
    \hline
    3. & Martie & 9. & Septembrie\\
    \hline
    4. & Aprilie & 10. & Octombrie\\
    \hline
    5. & Mai & 11. & Noiembrie\\
    \hline
    6. & Iunie & 12. & Decembrie\\
    \hline
  \end{tabular}
\end{center}
Für ersten Tag im Monat kann man sowohl unu als auch întâi sagen.
\begin{equation*}
  \text{Ce dată este/e azi?} =
  \begin{cases}
    \text{Astazi este/e 30 septembrie 2021}\\
    \text{Astazi este/e întâi/unu mai}\\
  \end{cases}
\end{equation*}
%
\textcolor{Red}{1}\textcolor{Green}{9}\textcolor{Cyan}{89} $\rightarrow$ \textcolor{Red}{O mie} \textcolor{Green}{nouă sute} \textcolor{Cyan}{optzeci și nouă}\\
2021 $\rightarrow$ Două mii douăzeci și unu\\
%
\subsection{Articul hotărât - Der Bestimmte Artikel}
Anders als im Deutschen wird im Rumänischen der bestimmte Artikel durch
anhängen von Buchstaben gebildet.\\
\newline
\begin{equation*}
  \text{Singular} =
  \begin{cases}
    \text{m. + n.} & \text{Mit Endung \textcolor{Red}{-l, -ul, -le}}\\
    \text{f.} & \text{Mit Endung \textcolor{Cyan}{-ua, -a}}\\
  \end{cases}
\end{equation*}
%
\begin{equation*}
  \text{Plural} =
  \begin{cases}
    \text{m.} & \text{Mit Endung \textcolor{Red}{-i}}\\
    \text{f. + n.} & \text{Mit Endung \textcolor{Cyan}{-le}}\\
  \end{cases}
\end{equation*}
Exemplu:\\
\begin{center}
  \begin{tabular}{ | R{3cm}||R{3cm}| } 
    \hline
    \multicolumn{2}{|c|}{Maskulin}\\
    \hline
    Un student & Doi studenți\\
    student\textcolor{Red}{ul} & studenți\textcolor{Red}{i}\\
    \hline
    Un leu & Doi lei\\
    leu\textcolor{Red}{l} & lei\textcolor{Red}{i}\\
    \hline
    Un frate & Doi frați\\
    frate\textcolor{Red}{le} & frați\textcolor{Red}{i}\\
    \hline
  \end{tabular}
\end{center}
%
\begin{center}
  \begin{tabular}{ | R{3cm}||R{3cm}| } 
    \hline
    \multicolumn{2}{|c|}{Feminin}\\
    \hline
    O studentă & Două studente\\
    student\textcolor{Cyan}{a} & studente\textcolor{Cyan}{le}\\
    \hline
    O carte & Două carți\\
    carte\textcolor{Cyan}{a} & carți\textcolor{Cyan}{le}\\
    \hline
    O familie & Două familii\\
    famili\textcolor{Cyan}{a} & familii\textcolor{Cyan}{le}\\
    \hline
    O cafea & Două cafele\\
    cafea\textcolor{Cyan}{ua} & cafele\textcolor{Cyan}{le}\\
    \hline
    O lalea & Două lalele\\
    lalea\textcolor{Cyan}{ua} & lalele\textcolor{Cyan}{le}\\
    \hline
  \end{tabular}
\end{center}
%
\begin{center}
  \begin{tabular}{ | R{3cm}||R{3cm}| } 
    \hline
    \multicolumn{2}{|c|}{Neutrum}\\
    \hline
    Un stilou & Două stilouri\\
    stilou\textcolor{Red}{l} & stilouri\textcolor{Cyan}{le}\\
    \hline
    Un scaun & Două scaune\\
    scaun\textcolor{Red}{ul} & scaune\textcolor{Cyan}{le}\\
    \hline
  \end{tabular}
\end{center}
%
Spezifiziert man Sachen näher so verwendet man den bestimmten Artikel.\\
\newline
Stau în hotel $\rightarrow$ Ich bleibe in einem Hotel\\
Stau în hotel\textcolor{Red}{ul continental} $\rightarrow$ Ich bleibe in dem 
Hotel Continental\\
\newline
Der bestimmte Artikel wird obligatorisch, bei \emph{cu + Verkehrsmittel/Werkzeug}\\
Merg \textcolor{Red}{cu} biciclet\textcolor{Green}{a}. $\rightarrow$ Ich fahre 
\textcolor{Red}{mit} \textcolor{Green}{dem} Fahrrad.\\
Mănânc supă \textcolor{Red}{cu} lingur\textcolor{Green}{a} $\rightarrow$ Ich esse 
Suppe \textcolor{Green}{mit} \textcolor{Red}{dem} Löffel.\\
Mai târziu merg acasă \textcolor{Red}{cu} mașin\textcolor{Red}{a} sau 
\textcolor{Green}{cu} autobuz\textcolor{Green}{ul}.\\
%
%
\subsection{Frăza și Vocabulă}
\begin{center}
  \begin{tabular}{ | p{6cm}| p{6cm} | } 
    \hline
    Un anotimp - Două anotimpuri & Jahreszeit\\
    \hline
    O primăvară Două primăveri & Frühling\\
    \hline
    O vară - Două veri & Sommer\\
    \hline
    O toamnă - Două toamne & Herbst\\
    \hline
    O iarnă - Două ierni & Winter\\
    \hline
    O mie - Două mii & Tausend\\
    \hline
    O lună - Două luni & Monat\\
    \hline
    De pe \dots până pe \dots & Von (Datum) bis (Datum)\\
    \hline
    Târziu & Spät\\
    \hline
    Devreme & Früh\\
    \hline
  \end{tabular}
\end{center}
%
\newpage
\section{Lecție}

\subsection{Forma de politețe - Höflichkeitsform}
%
\begin{center}
  \begin{tabular}{ |p{3cm}|p{3cm}||p{3cm}|p{3cm}|  }
      \hline
      Eu aș dori & Ich hätte gerne & Noi am dori & Wir hätten gerne\\
      \hline
      Tu ai dori & Du hättest gerne & Voi ați dori & Ihr hättet gerne\\
      \hline
      El ar dori& Er hätte gerne & Ei ar dori & Sie hätten gerne\\ 
      Ea ar dori & Sie hätte gerne & Ele ar dori & Sie hätten gerne\\
      \hline
     \end{tabular}
\end{center}
Eu aș dori o bere vă rog. $\rightarrow$ Ich hätte gerne ein Bier, bitte.\\
Prietena mea ar dori un vin vă rog. $\rightarrow$ Meine Freundin hätte gerne einen Wein, bitte.\\
Ce ați dori? $\rightarrow$ Was hätten Sie gerne?\\
\newline
Wir können diese Form aber auch mit anderen Verben als \emph{a dori} 
verwenden um Konjunktiv-Sätze zu bilden.\\
\newline
Ar fi frumos, dacă aș fi la tine. $\rightarrow$ Es wäre schön, wenn ich bei dir wäre.
%
\subsection{Să}
Das \emph{să} wird als "Bindemittel" zwischen zwei Verben benutzt.\\
\newline
Doresc să merg. $\rightarrow$ Ich möchte gehen.\\
\textcolor{Red}{Doresc} să \textcolor{Green}{mergeți} la școală $\rightarrow$ \textcolor{Red}{Ich will}, dass \textcolor{Green}{Ihr} in die Schule \textcolor{Green}{geht}.\\
Vrem să mergem la universitate? $\rightarrow$ Wollen wir zur Uni gehen?
%
\subsection{Frăza și Vocabulă}
\begin{center}
  \begin{tabular}{ | p{6cm}| p{6cm} | } 
    \hline
    Nota de plată & Rechnung\\
    \hline
    O sticlă - Două sticle & Flasche\\
    \hline
    Plin(ă) & Voll\\
    \hline
    Gol (Goală) & Leer\\
    \hline
    O friptură - Două fripture & Braten\\
    \hline
    O virgulă - Două virgule & Komma\\
    \hline
    Aici & Hier\\
    \hline
    Acolo & Dort\\
    \hline
    Sus & Oben\\
    \hline
    Jos & Unten\\
    \hline
    Aproape & Nah\\
    \hline
    Departe & Weit\\
    \hline
    În stânga & Links\\
    \hline
    În dreapta & Rechts\\
    \hline
  \end{tabular}
\end{center}
\begin{center}
  \begin{tabular}{ | p{6cm}| p{6cm} | } 
    \hline
    În față & Vorne\\
    \hline
    În spate & Hinten\\
    \hline
  \end{tabular}
\end{center}

\section{Lecție}
\subsection{Reflexive Verben}
Reflexive Verben werden wie folgt gebildet. Die in rot markierten Formen
sind bei allen reflexiven Verben gleich.
%
\begin{center}
  \begin{tabular}{ |p{3.25cm}|p{3.25cm}||p{3.25cm}|p{3.25cm}|  }
      \hline
      Eu \textcolor{Red}{mă} trezesc & Ich wache auf & Noi \textcolor{Red}{ne} trezim & Wir wachen auf\\
      \hline
      Tu \textcolor{Red}{te} trezești & Du wachst auf & Voi \textcolor{Red}{vă} trezeți & Ihr wacht auf\\
      \hline
      El \textcolor{Red}{se} trezește & Er wacht auf & Ei \textcolor{Red}{se} trezesc & Sie wachen auf\\ 
      Ea \textcolor{Red}{se} trezește & Sie wacht auf & Ele \textcolor{Red}{se} tresesc & Sie wachen auf\\
      \hline
     \end{tabular}
\end{center}
%
%
\subsection{Trecut - Vergangenheit}
Die Vergangenheit bildet sich durch \emph{a avea + prticipiu}. Die a
Form \emph{a avea} ist dabei etwas abgeändert. Alle Verben werden auf 
diese Art gebildet. Die Partizip-Form bleibt dabei immer die selbe 
und nur die Form von \emph{a avea} wird konjugiert.
%
\begin{center}
  \begin{tabular}{ |p{3.25cm}|p{3.25cm}||p{3.25cm}|p{3.25cm}|  }
      \hline
      Eu am facut & Ich habe gemacht & Noi am facut & Wir haben gemacht\\
      \hline
      Tu ai facut & Du hast gemacht & Voi ați facut & Ihr habt gemacht\\
      \hline
      El a facut & Er hat gemacht & Ei au facut & Sie haben gemacht\\ 
      Ea a facut & Sie hat gemacht& Ele au facut & Sie haben gemacht\\
      \hline
     \end{tabular}
\end{center}
%
%
Es gibt Regeln für das Partizip. Verben die auf \emph{a, i und î} enden, 
wird ein \emph{t} angehangen. Bei der Endung \emph{ea} wird \emph{ut} 
angehangen. Verben die auf \emph{e} enden sind oft auch unregelmäßig 
(a merge $\rightarrow$ mers).
\begin{itemize}
  \item -a, -i, -î $\rightarrow$ +t
  \item -ea $\rightarrow$ +ut 
  \item -e $\rightarrow$ +s, +t, +ut
\end{itemize}

\subsection{Vergangenheit + Reflexive Verben}
Verwendet man ein reflexives Verb in der Vergangenheit, werden die Formen 
\emph{a avea} und die Form von \emph{se} etwas verkürzt bzw. anders 
geschrieben.
%
\begin{center}
  \begin{tabular}{ |p{3.25cm}|p{3.25cm}||p{3.25cm}|p{3.25cm}|  }
      \hline
      Eu m-am trezit & Ich bin aufgewacht & Noi ne-am trezit & Wir sind aufgewacht\\
      \hline
      Tu te-ai trezit & Du bist aufgewacht & Voi v-ați trezit & Ihr seid aufgewacht\\
      \hline
      El s-a trezit & Er ist aufgewacht & Ei s-au trezit & Sie sind aufgewacht\\ 
      Ea s-a trezit & Sie ist aufgewacht & Ele s-au trezit & Sie sind aufgewacht\\
      \hline
     \end{tabular}
\end{center}

\begin{center}
  \begin{tabular}{ | p{6cm}| p{6cm} | } 
    \hline
    a se scula & aufstehen\\
    \hline
    a se trezi-esc & aufwachen\\
    \hline
    a se spăla & sich waschen\\
    \hline
    a se îmbrăca & sich anziehen\\
    \hline
    a se debrăca & sich ausziehen\\
    \hline
    a se grabi-esc & sich beeilen\\
    \hline
    a gândi-esc & denken\\
    \hline
    ceva & etwas\\
    \hline
    pentru că & weil\\
    \hline
    că & dass, weil, denn\\
    \hline
    o ciupercă - Două ciuperci & Pilz\\
    \hline
    adevărat & Richtig\\
    \hline
    fals & Flasch\\
    \hline
    a zice & sagen\\
    \hline
    a spune & sagen\\
    \hline
  \end{tabular}
\end{center}

\section{Lecție}

\subsection{Conjunctiv Prezent - Konjunktiv Präsens}
Doresc să merg la Paris $\rightarrow$ Ich will nach Paris gehen.\\
Eu pot să citesc ziare multe $\rightarrow$ Ich kann viele Zeitungen lesen.\\
Vreau să beau o cafea $\rightarrow$ Ich will einen Kaffee mit Milch trinken.\\
Trebuie să vii acasă $\rightarrow$ Du musst nach Hause kommen.\\
Nu ai voie să fumesc în casă $\rightarrow$ Du darfst nicht im Haus rauchen.\\
Știu să gâtesc $\rightarrow$ Ich kann kochen.\\
\newline
In jeweils der 3. Person Singular und Plural sind die Formen des Konjunktiv 
abgeändert.

\begin{center}
  \begin{tabular}{ |p{3.25cm}|p{3.25cm}||p{3.25cm}|p{3.25cm}|  }
      \hline
      Eu vreau să merg acasă & Ich will nach Hause gehen & Noi vrem să mergem acasă & Wir wollen nach Hause gehen\\
      \hline
      Tu vrei să mergi acasă & Tu willst nach Hause gehen & Voi vreți să mergeți acasă & Ihr wollt nach Hause gehen\\
      \hline
      El vrea să m\textcolor{Red}{ea}rg\textcolor{Red}{ă} acasă & Er will nach Hause gehen & Ei vor să m\textcolor{Red}{ea}rg\textcolor{Red}{ă} & Sie wollen nach Hause gehen\\ 
      Ea vrea să m\textcolor{Red}{ea}rg\textcolor{Red}{ă} acasă & Sie will nach Hause gehen & Ele vor să m\textcolor{Red}{ea}rg\textcolor{Red}{ă} & Sie wollen nach Hause gehen\\
      \hline
     \end{tabular}
\end{center}
Endet das Verb in der 3. Person Präsens auf \emph{ă}, so wird daraus im Konjunktiv 
ein \emph{e}. Enthält ein Verb im Präsens vor der Endung noch ein \emph{ea} 
(lucrează, pleacă, \dots), so wird das \emph{ea} auch durch ein \emph{e} ersetzt.\\
\newline
El/Ea mănânc\textcolor{Red}{ă} $\rightarrow$ să mănânc\textcolor{Red}{e}\\
El/Ea lucr\textcolor{Green}{ea}z\textcolor{Red}{ă} $\rightarrow$ să lucr\textcolor{Green}{e}z\textcolor{Red}{e}\\
El/Ea pl\textcolor{Green}{ea}c\textcolor{Red}{ă} $\rightarrow$ să pl\textcolor{Green}{e}c\textcolor{Red}{e}\\
\newline
Endet ein Verb in der 3. Person Präsens auf \emph{e}, kann die Regel genau umgekehrt 
angewandt werden. Im Konjunktiv wird aus dem \emph{e} ein \emph{ă} und steht vor dem 
\emph{e} ein weiteres \emph{e} so wird dies zu einem \emph{ea}.\\
\newline
El/Ea vin\textcolor{Red}{e} $\rightarrow$ să vin\textcolor{Red}{ă}\\
El/Ea cit\textcolor{Green}{e}șt\textcolor{Red}{e} $\rightarrow$ să cit\textcolor{Green}{ea}sc\textcolor{Red}{ă}\\
El/Ea vorb\textcolor{Green}{e}șt\textcolor{Red}{e} $\rightarrow$ să vorb\textcolor{Green}{ea}sc\textcolor{Red}{ă}\\
El/Ea înc\textcolor{Green}{e}p\textcolor{Red}{e} $\rightarrow$ să înc\textcolor{Green}{ea}p\textcolor{Red}{ă}\\
El/Ea m\textcolor{Green}{e}rg\textcolor{Red}{e} $\rightarrow$ să m\textcolor{Green}{ea}rg\textcolor{Red}{ă}
%
\subsubsection{Excepții - Ausnahmen}
El ia (Er nimmt) $\rightarrow$ vrea să ia (Er will nehmen)\\
El bea (Er trinkt) $\rightarrow$ vrea să bea (Er will trinken)\\
El scrie (Er schreibt) $\rightarrow$ vrea să scrie (Er will schreiben)\\
El stă (Er bleibt) $\rightarrow$ vrea să stea (Er will bleiben)\\
El dă (Er gibt) $\rightarrow$ vrea să dea (Er will geben)\\
\newline
Ausschließlich mit \emph{a putea} kann man den Konjunktiv auch mit dem Infinitiv bilden. 
\begin{equation*}
  \text{a putea} =
  \begin{cases}
    \text{+ să + konjunktiv} & \text{Eu pot să merg în parc} \\
    \text{+ infinitiv} & \text{Eu pot merge în parc}\\
  \end{cases}
\end{equation*}

\begin{center}
  \begin{tabular}{ |p{3.25cm}|p{3.25cm}||p{3.25cm}|p{3.25cm}|  }
      \hline  
      \multicolumn{4}{|c|}{a fi \& a avea} \\
      \hline
      \hline
      Eu să fiu & Noi să fim & Eu să am & Noi să avem\\
      \hline
      Tu să fii & Voi să fiți & Tu să ai & Voi să aveți\\
      \hline
      El să fie& Ei să fie & El să aibă & Ei să aibă\\ 
      Ea să fie& Ele să fie & Ea să aibă & Ele să aibă\\
      \hline
     \end{tabular}
\end{center}

%\subsection{Verbi}
%\begin{center}
%  \begin{tabular}{ | p{6cm}| p{6cm} | } 
%    \hline
%    După aceea & Später\\
%    \hline
%    A avea voie & Dürfen\\
%
%    \hline
%  \end{tabular}
%\end{center}

\pagebreak
\section{Lecție}
\subsection{Adverbe de timp - Zeitadverbien}
\subsubsection*{Niciodată - Nie}
Bei Niciodată muss \emph{immer} auch ein nu verwendet werden. Man muss also quasi 
doppelt verneinen, wenn man ausdrücken will, dass man etwas \emph{nie} macht.\\
\emph{Niciodată nu am fost la Paris $\rightarrow$ Ich war noch nie in Paris.}\\
%
\subsubsection*{Restul - Alles andere}
Rar = Selten $\rightarrow$ Merg rar la cinema.\\
Uneori = Manchmal $\rightarrow$ Uneori merg în discotecă.\\
Des = Oft $\rightarrow$ Merg des la teatru.\\
De obicei = Gewöhnlich/Normalerweise $\rightarrow$ De obicei merg la cumpărături cu prietena mea.\\
Întotdeauna/Mereu = Immer $\rightarrow$ Fac întotdeauna sport în week-end.
\subsection{Propoziții - Sätze}
Puteți să îmi aduceți meniul, vă rog? $\rightarrow$ Können Sie mir bitte das Menu bringen?\\
Ce îmi puteți recomenda? $\rightarrow$ Was können Sie mir empfehlen?\\
Ce îmi recomendați? $\rightarrow$ Was empfehlen Sie mir?\\
Aveți o listă de vinuri? $\rightarrow$ Haben Sie eine Weinkarte?\\
Nu mulțumesc, este/e suficient. $\rightarrow$ Nein Danke, das reicht.\\
Mâncare este/e foarte bună. $\rightarrow$ Das Essen ist sehr gut.\\
Îmi place foarte mult. $\rightarrow$ Es schmeckt mir sehr.\\
Aș dori să rezerv o masă pe numele \dots $\rightarrow$ Ich würde gerne einen Tisch auf den Namen \dots reservieren.\\
Am rezervat o masă pe numele \dots $\rightarrow$ Ich habe einen Tisch auf den Namen \dots reserviert.
\pagebreak
\section{Lecție}
\subsection{Adjectivul calificativ - Adjektive}
Adjektive beschreiben etwas näher (warm, kalt, groß, klein, \dots). 
Adektive lassen sich in 4 Gruppen einteilen.\\
%
\subsubsection*{Gruppe 1}
Adjektive in Gruppe 1 sind \emph{immer} gleich. Dabei ist es egal in welcher Form sich das 
Wort, auf welches sich das Adjektiv bezieht, befindet.
\begin{center}
  \begin{tabular}{ | C{3cm}| C{3cm} | C{3cm} | C{3cm} |}
    \hline
    \multicolumn{2}{|c|}{Singular} & \multicolumn{2}{c|}{Plural}\\
    \hline 
    m. + n. & f. & m. & f. + n.\\
    \hline
    \hline
    \multicolumn{4}{|c|}{gri}\\
    \hline
    \multicolumn{4}{|c|}{mov}\\
    \hline
    \multicolumn{4}{|c|}{maro}\\
    \hline
  \end{tabular}
\end{center}
%
%
\subsubsection*{Gruppe 2}
Adjektive der Gruppe 2 haben 2 Formen. Eine für Singular und eine für den Plural.
Das Geschlecht ist dabei egal. Häufig sind Adjektive welche auf \emph{-e} enden 
in der zweiten Gruppe.
\begin{center}
  \begin{tabular}{ | C{3cm}| C{3cm} | C{3cm} | C{3cm} |}
    \hline
    \multicolumn{2}{|c|}{Singular} & \multicolumn{2}{c|}{Plural}\\
    \hline 
    m. + n.& f.& m. & f. + n.\\
    \hline
    \hline
    \multicolumn{2}{|c|}{verde} & \multicolumn{2}{c|}{verzi}\\
    \hline
    \multicolumn{2}{|c|}{rece} & \multicolumn{2}{c|}{reci}\\
    \hline
    \multicolumn{2}{|c|}{dulce} & \multicolumn{2}{c|}{dulci}\\
    \hline
  \end{tabular}
\end{center}
%
%
\subsubsection*{Gruppe 3}
Adjektive der Gruppe 3 haben 3 Formen. Im Singular passt sich das Adjektiv an 
das Wort auf welches es sich bezieht an und im Plural gibt es nur eine Form.
\begin{center}
  \begin{tabular}{ | C{3cm}| C{3cm} | C{3cm} | C{3cm} |}
    \hline
    \multicolumn{2}{|c|}{Singular} & \multicolumn{2}{c|}{Plural}\\
    \hline 
    m. + n.& f.& m. & f. + n.\\
    \hline
    \hline
    roșu & roșie & \multicolumn{2}{c|}{roșii}\\
    \hline
    larg & largă & \multicolumn{2}{c|}{largi}\\
    \hline
    lung & lungă & \multicolumn{2}{c|}{lungi}\\
    \hline
  \end{tabular}
\end{center}
%
%
\subsubsection*{Gruppe 4}
Adjektive der Gruppe 4 passen sich an die Form des Wortes auf welches es sich 
bezieht an.
\begin{center}
  \begin{tabular}{ | C{3cm}| C{3cm} | C{3cm} | C{3cm} |}
    \hline
    \multicolumn{2}{|c|}{Singular} & \multicolumn{2}{c|}{Plural}\\
    \hline 
    m. + n.& f.& m. & f. + n.\\
    \hline
    \hline
    albastru & albastră & albaștri & albastre\\
    \hline
    cald & caldă & calzi & calde\\
    \hline
    colorat & colorată & colorați & colorate\\
    \hline
  \end{tabular}
\end{center}
%
%
\subsection{Ajectivul posesiv - Possesivpronomen}
Wenn ein Hauptwort von einem Possesivpronomen (mein, dein, sein, \dots) begleitet wird 
so muss das Hauptwort mit dem Bestimmten Artikel benutzt werden.
%
\begin{center}
  \begin{tabular}{ | C{1.7cm}| C{1.7cm} | C{1.7cm} | C{1.7cm} | C{1.7cm} | C{1.7cm} | C{1.7cm} |}
    \hline
              & mein & dein & sein(e) & unser & euer & ihr\\
              & meine & deine & ihr(e) & unsere & eure & ihre\\
    \hline
    pantof\textcolor{Red}{ul}  & me\textcolor{Red}{u} & tă\textcolor{Red}{u} & lui/ei & nostr\textcolor{Red}{u} & vostr\textcolor{Red}{u} & lor\\
    \hline
    cămaș\textcolor{Cyan}{a} & me\textcolor{Cyan}{a} & t\textcolor{Cyan}{a} & lui/ei & noastr\textcolor{Cyan}{ă} & voastr\textcolor{Cyan}{ă} & lor\\
    \hline
    pantofi\textcolor{Red}{i} & me\textcolor{Red}{i} & tă\textcolor{Red}{i} & lui/ei & noștr\textcolor{Red}{i} & voștr\textcolor{Red}{i} & lor\\
    \hline
    cămăși\textcolor{Cyan}{le} & me\textcolor{Cyan}{le} & ta\textcolor{Cyan}{le} & lui/ei & noastr\textcolor{Cyan}{e} & voastr\textcolor{Cyan}{e} & lor\\
    \hline
  \end{tabular}
\end{center}




\subsection{Vocabulă}
\subsubsection*{Substantive}
\begin{center}
  \begin{tabular}{ | p{6cm}| p{6cm} | } 
    \hline
    agenție de turism (f.) & Reisebüro\\
    \hline
    bagaj, bagaje (n.) & Gepäck\\
    \hline
    bluză, bluze (f.) & Bluse\\
    \hline
    cazare, cazări (f.) & Unterkunft\\
    \hline
    cămașă, cămăși (f.) & Hemd\\
    \hline
    concediu, concedii (n.) & Urlaub\\
    \hline
    costum, costume (n.) & Anzug\\
    \hline
    costum de baie (n.) & Badeanzug\\
    \hline
    cravată, cravate (f.) & Krawatte\\
    \hline
    culoare, culori (f.) & Farbe\\
    \hline
    curte, curți (f.) & Hof\\
    \hline
    emisiune, emisiuni (f.) & Sendung\\
    \hline
    floare, flori (f.) & Blume\\
    \hline
    fustă, fuste (f.) & Rock\\
    \hline
    geacă, geci (f.) & Jacke\\
    \hline
    geantă, genți (f.) & Tasche\\
    \hline
    grădină, grădini (f.) & Garten\\
    \hline
    haină, haine (f.) & Kleider\\
    \hline
    îmbrăcăminte (f.) & Kleidung\\
    \hline
    lenjerie, lenjerii (f.) & Unterwäsche\\
    \hline
    mare, mări (f.) & Meer\\
    \hline
    motiv, motive (n.) & Grund\\
    \hline
    pantalon, pantaloni (m.) & Hose\\
    \hline
    pantaloni scurți & Kurze Hose\\
    \hline
    pantof, pantofi (m.) & Schuhe\\
    \hline
    pardesiu, pardesie (n.) & Mantel\\
    \hline
    pădure, păduri (f.) & Wald\\
    \hline
    pulover, pulovere (n.) & Pullover\\
    \hline
    rochie, rochii (f.) & Kleid\\
    \hline
    sacou, sacouri (n.) & Sakko\\
    \hline
    scaun, scaune (n.) & Stuhl\\
    \hline
    spectacol, spectacole (n.) & Vorstellung\\
    \hline
    șosetă, șosete (f.) & Socken\\
    \hline
    tricou, tricouri (n.) & T-Shirt\\
    \hline
    zgomot, zgomote (n.) & Geräusch\\
    \hline
  \end{tabular}
\end{center}
%
%
\subsubsection*{Alte părți de vorbire, expresii}
\begin{center}
  \begin{tabular}{ | p{6cm}| p{6cm} | } 
    \hline
    albastru, -ă, -ștri, -e & Blau\\
    \hline
    bej & Beige\\
    \hline
    cald, -ă, -zi, -e & Warm\\
    \hline
    colorat, -ă, -ți, -e & Bunt\\
    \hline
    cuminte, -ți & Brav\\
    \hline
    dificil, -ă, -i, -e & Schwierig\\
    \hline
    dreapta & Rechts\\
    \hline
    dulce, -i & Süß\\
    \hline
    elegant, -ă, -ți, -e & Elegant\\
    \hline
    frumos, -oasă, -și, -oase & Schön\\
    \hline
    galben, -ă, -i, -e & Gelb\\
    \hline
    greu, grea, grei, grele & Schwer\\
    \hline
    gri & Grau\\
    \hline
    în plus & Außerdem\\
    \hline
    înalt, -ă, -ți, -e & Hoch, Groß\\
    \hline
    interesant, -ă, -ți, -e & Interessant\\
    \hline
    larg, -ă, -i & Breit\\
    \hline
    luminos, -oasă, -și, -oase & Hell\\
    \hline
    lung, -ă, -i & Lang\\
    \hline
    mare, -i & Groß\\
    \hline
    maro & Braun\\
    \hline
    mic, -ă, -i & Klein\\
    \hline
    minunat, -ă, -ți, -e & Wunderbar\\
    \hline
    mov & Lila\\
    \hline
    negru, neagră, negri, negre & Schwarz\\
    \hline
    nou, -ă, -i & Neu\\
    \hline
    plăcut, -ă, -ți, -e & Angenehm\\
    \hline
    portocaliu, -e, -i & Orange\\
    \hline
    preferat, -ă, -ți, -e & Lieblings-\\
    \hline
    proaspăt, -ă, -eți, -ete & Frisch\\
    \hline
    rău, rea, răi, rele & Schlimm/Böse\\
    \hline
    roz & Rosa\\
    \hline
    sigur, -ă, -i, -e & Sicher\\
    \hline
    special, -ă, -i, -e & Speziell\\
    \hline
    stânga & Links\\
    \hline
    urât, -ă, -ți, -e & Hässlich\\
    \hline
    vechi, veche & Alt\\
    \hline
    verde, verzi & Grün\\
    \hline
  \end{tabular}
\end{center}

%
%
%
%
%
%
%
%
%
\singlespacing
\newpage
\section{Verbi}
Die Form \emph{ele} wird bei Gruppen benutzt, die \underline{nur} aus 
Frauen besteht. Sobald ein Mann dabei ist verwendet man \emph{ei}. 
Die Höflichkeitsform ist die gleiche Form wie der 2. Plural (voi) aber 
mit \emph{Dumneavoastră}. Also Beispielsweise: Dumneavoastră sunteți
\index{a fi}
\begin{center}
  \begin{tabular}{ |p{3.25cm}|p{3.25cm}||p{3.25cm}|p{3.25cm}|  }
      \hline
      \multicolumn{4}{|c|}{a fi - zu sein} \\
      \hline
      \hline
      Eu sunt & Ich bin & Noi suntem & Wir sind\\
      \hline
      Tu ești & Du bist & Voi sunteți & Ihr seid\\
      \hline
      El este/e & Er ist & Ei sunt & Sie sind\\ 
      Ea este/e & Sie ist & Ele sunt & Sie sind\\
      \hline
      \multicolumn{4}{|c|}{participiu: fost} \\
      \hline
     \end{tabular}
\end{center}
%
\index{a avea}
\begin{center}
  \begin{tabular}{ |p{3.25cm}|p{3.25cm}||p{3.25cm}|p{3.25cm}| }
      \hline
      \multicolumn{4}{|c|}{a avea - zu haben} \\
      \hline
      \hline
      Eu am & Ich habe & Noi avem & Wir haben\\
      \hline
      Tu ai & Du hast & Voi aveți & Ihr habt\\
      \hline
      El are & Er hat & Ei au & Sie haben\\ 
      Ea are & Sie hat & Ele au & Sie haben\\
      \hline
      \multicolumn{4}{|c|}{participiu: avut} \\
      \hline
     \end{tabular}
\end{center}
%
\index{a merge}
\begin{center}
  \begin{tabular}{ |p{3.25cm}|p{3.25cm}||p{3.25cm}|p{3.25cm}| }
      \hline
      \multicolumn{4}{|c|}{a merge - zu gehen} \\
      \hline
      \hline
      Eu merg & Ich gehe & Noi mergem & Wir gehen\\
      \hline
      Tu mergi & Du gehst & Voi mergeți & Ihr geht\\
      \hline
      El merge & Er geht & Ei merg & Sie gehen\\ 
      Ea merge & Sie geht & Ele merg & Sie gehen\\
      \hline
      \multicolumn{4}{|c|}{participiu: mers} \\
      \hline
     \end{tabular}
\end{center}
%
\index{a putea}
\begin{center}
  \begin{tabular}{ |p{3.25cm}|p{3.25cm}||p{3.25cm}|p{3.25cm}| }
      \hline
      \multicolumn{4}{|c|}{a putea - zu können} \\
      \hline
      \hline
      Eu pot & Ich kann & Noi putem & Wir können\\
      \hline
      Tu poți & Du kannst & Voi puteți & Ihr könnt\\
      \hline
      El poate & Er kann & Ei pot & Sie können\\ 
      Ea poate & Sie kann & Ele pot & Sie können\\
      \hline
      \multicolumn{4}{|c|}{participiu: putut} \\
      \hline
     \end{tabular}
\end{center}
%
\index{a face}
\begin{center}
  \begin{tabular}{ |p{3.25cm}|p{3.25cm}||p{3.25cm}|p{3.25cm}| }
      \hline
      \multicolumn{4}{|c|}{a face - zu machen} \\
      \hline
      \hline
      Eu fac & Ich mache & Noi facem & Wir machen\\
      \hline
      Tu faci & Du machst & Voi faceți & Ihr macht\\
      \hline
      El face & Er macht & Ei fac & Sie machen\\ 
      Ea face & Sie macht & Ele fac & Sie machen\\
      \hline
      \multicolumn{4}{|c|}{participiu: facut} \\
      \hline
     \end{tabular}
\end{center}
%
\index{a vrea}
\begin{center}
  \begin{tabular}{ |p{3.25cm}|p{3.25cm}||p{3.25cm}|p{3.25cm}| }
      \hline
      \multicolumn{4}{|c|}{a vrea - zu wollen} \\
      \hline
      \hline
      Eu vreau & Ich will & Noi vrem & Wir wollen\\
      \hline
      Tu vrei & Du willst & Voi vreți & Ihr wollt\\
      \hline
      El vrea & Er will & Ei vor & Sie wollen\\ 
      Ea vrea & Sie will & Ele vor & Sie wollen\\
      \hline
      \multicolumn{4}{|c|}{participiu: vrut} \\
      \hline
     \end{tabular}
\end{center}
%
\index{a da}
\begin{center}
  \begin{tabular}{ |p{3.25cm}|p{3.25cm}||p{3.25cm}|p{3.25cm}| }
      \hline
      \multicolumn{4}{|c|}{a da - zu geben} \\
      \hline
      \hline
      Eu dau & Ich gebe & Noi dăm & Wir geben\\
      \hline
      Tu dai & Du gibst & Voi dați & Ihr gebt\\
      \hline
      El dă & Er gibt & Ei dau & Sie geben\\ 
      Ea dă & Sie gibt & Ele dau & Sie geben\\
      \hline
      \multicolumn{4}{|c|}{participiu: dat} \\
      \hline
     \end{tabular}
\end{center}
%
\index{a lucra-ez}
\begin{center}
  \begin{tabular}{ |p{3.25cm}|p{3.25cm}||p{3.25cm}|p{3.25cm}| }
      \hline
      \multicolumn{4}{|c|}{a lucra-ez - zu arbeiten} \\
      \hline
      \hline
      Eu lucrez & Ich arbeite & Noi lucrăm & Wir arbeiten\\
      \hline
      Tu lucrezi & Du arbeitest & Voi lucrați & Ihr arbeitet\\
      \hline
      El lucrează & Er arbeitet & Ei lucrează & Sie arbeiten\\ 
      Ea lucrează & Sie arbeitet & Ele lucrează & Sie arbeiten\\
      \hline
      \multicolumn{4}{|c|}{participiu: lucrat} \\
      \hline
     \end{tabular}
\end{center}
%
\index{a lua}
\begin{center}
  \begin{tabular}{ |p{3.25cm}|p{3.25cm}||p{3.25cm}|p{3.25cm}| }
      \hline
      \multicolumn{4}{|c|}{a lua - zu nehmen} \\
      \hline
      \hline
      Eu iau & Ich nehme & Noi luăm & Wir nehmen\\
      \hline
      Tu iei & Du nimmst & Voi luați & Ihr nehmt\\
      \hline
      El ia & Er nimmt & Ei iau & Sie nehmen\\ 
      Ea ia & Sie nimmt & Ele iau & Sie nehmen\\
      \hline
      \multicolumn{4}{|c|}{participiu: luat} \\
      \hline
     \end{tabular}
\end{center}
%
\index{a pleca}
\begin{center}
  \begin{tabular}{ |p{3.25cm}|p{3.25cm}||p{3.25cm}|p{3.25cm}| }
      \hline
      \multicolumn{4}{|c|}{a pleca - zu verlassen} \\
      \hline
      \hline
      Eu plec & Ich verlasse & Noi plecăm & Wir verlassen\\
      \hline
      Tu pleci & Du verlässt & Voi plecați & Ihr verlasst\\
      \hline
      El pleacă & Er verlässt & Ei pleacă & Sie verlassen\\ 
      Ea pleacă & Sie verlässt & Ele pleacă & Sie verlassen\\
      \hline
      \multicolumn{4}{|c|}{participiu: plecat} \\
      \hline
     \end{tabular}
\end{center}
%
\index{a sta}
\begin{center}
  \begin{tabular}{ |p{3.25cm}|p{3.25cm}||p{3.25cm}|p{3.25cm}| }
      \hline
      \multicolumn{4}{|c|}{a sta - zu bleiben} \\
      \hline
      \hline
      Eu stau & Ich bleibe & Noi stăm & Wir bleiben\\
      \hline
      Tu stai & Du bleibst & Voi stați & Ihr bleibt\\
      \hline
      El stă & Er bleibt & Ei stau & Sie bleiben\\ 
      Ea stă & Sie bleibt & Ele stau & Sie bleiben\\
      \hline
      \multicolumn{4}{|c|}{participiu: stat} \\
      \hline
     \end{tabular}
\end{center}
%
\index{a începe}
\begin{center}
  \begin{tabular}{ |p{3.25cm}|p{3.25cm}||p{3.25cm}|p{3.25cm}| }
      \hline
      \multicolumn{4}{|c|}{a începe - anfangen} \\
      \hline
      \hline
      Eu încep & Ich fange an & Noi începem & Wir fangen an\\
      \hline
      Tu începi & Du fängst an & Voi începeți & Ihr fangt an\\
      \hline
      El începe & Er fängt an & Ei încep & Sie fangen an\\ 
      Ea începe & Sie fängt an & Ele încep & Sie fangen an\\
      \hline
      \multicolumn{4}{|c|}{participiu: început} \\
      \hline
     \end{tabular}
\end{center}
%
\index{a scrie}
\begin{center}
  \begin{tabular}{ |p{3.25cm}|p{3.25cm}||p{3.25cm}|p{3.25cm}| }
      \hline
      \multicolumn{4}{|c|}{a scrie - zu schreiben} \\
      \hline
      \hline
      Eu scriu & Ich schreibe & Noi scriem & Wir schreiben\\
      \hline
      Tu scrii & Du schreibst & Voi scrieți & Ihr schreibt\\
      \hline
      El scrie & Er schreibt & Ei scriu & Sie schreiben\\ 
      Ea scrie & Sie schreibt & Ele scriu & Sie schreiben\\
      \hline
      \multicolumn{4}{|c|}{participiu: scris} \\
      \hline
     \end{tabular}
\end{center}
%
\index{a vedea}
\begin{center}
  \begin{tabular}{ |p{3.25cm}|p{3.25cm}||p{3.25cm}|p{3.25cm}| }
      \hline
      \multicolumn{4}{|c|}{a vedea - zu sehen} \\
      \hline
      \hline
      Eu văd & Ich sehe & Noi vedem & Wir sehen\\
      \hline
      Tu vezi & Du siehst & Voi vedeți & Ihr seht\\
      \hline
      El vede & Er sieht & Ei văd & Sie sehen\\ 
      Ea vede & Sie sieht & Ele văd & Sie sehen\\
      \hline
      \multicolumn{4}{|c|}{participiu: văzut} \\
      \hline
     \end{tabular}
\end{center}
%
\index{a citi-esc}
\begin{center}
  \begin{tabular}{ |p{3.25cm}|p{3.25cm}||p{3.25cm}|p{3.25cm}| }
      \hline
      \multicolumn{4}{|c|}{a citi-esc - zu lesen} \\
      \hline
      \hline
      Eu citesc & Ich lese & Noi citim & Wir lesen\\
      \hline
      Tu citești & Du liest & Voi citiți & Ihr lest\\
      \hline
      El citește & Er liest & Ei citesc & Sie lesen\\ 
      Ea citește & Sie liest & Ele citesc & Sie lesen\\
      \hline
      \multicolumn{4}{|c|}{participiu: citit} \\
      \hline
     \end{tabular}
\end{center}
%
\index{a vorbi-esc}
\begin{center}
  \begin{tabular}{ |p{3.25cm}|p{3.25cm}||p{3.25cm}|p{3.25cm}| }
      \hline
      \multicolumn{4}{|c|}{a vorbi-esc - zu sprechen} \\
      \hline
      \hline
      Eu vorbesc & Ich spreche & Noi vorbim & Wir sprechen\\
      \hline
      Tu vorbești & Du sprichst & Voi vorbiți & Ihr sprecht\\
      \hline
      El vorbește & Er spricht & Ei vorbesc & Sie sprechen\\ 
      Ea vorbește & Sie spricht & Ele vorbesc & Sie sprechen\\
      \hline
      \multicolumn{4}{|c|}{participiu: vorbit} \\
      \hline
     \end{tabular}
\end{center}
%
\index{a veni}
\begin{center}
  \begin{tabular}{ |p{3.25cm}|p{3.25cm}||p{3.25cm}|p{3.25cm}| }
      \hline
      \multicolumn{4}{|c|}{a veni - zu kommen} \\
      \hline
      \hline
      Eu vin & Ich komme & Noi venim & Wir kommen\\
      \hline
      Tu vii & Du kommst & Voi veniți & Ihr kommt\\
      \hline
      El vine & Er kommt & Ei vin & Sie kommen\\ 
      Ea vine & Sie kommt & Ele vin & Sie kommen\\
      \hline
      \multicolumn{4}{|c|}{participiu: venit} \\
      \hline
     \end{tabular}
\end{center}
%
\index{a dori-esc}
\begin{center}
  \begin{tabular}{ |p{3.25cm}|p{3.25cm}||p{3.25cm}|p{3.25cm}| }
      \hline
      \multicolumn{4}{|c|}{a dori-esc - zu wünschen} \\
      \hline
      \hline
      Eu doresc & Ich wünsche & Noi dorim & Wir wünschen\\
      \hline
      Tu dorești & Du wünscht & Voi doriți & Ihr wünscht\\
      \hline
      El dorește & Er wünscht & Ei doresc & Sie wünschen\\ 
      Ea dorește & Sie wünscht & Ele doresc & Sie wünschen\\
      \hline
      \multicolumn{4}{|c|}{participiu: dorit} \\
      \hline
     \end{tabular}
\end{center}
%
\index{a semna-ez}
\begin{center}
  \begin{tabular}{ |p{3.25cm}|p{3.25cm}||p{3.25cm}|p{3.25cm}| }
      \hline
      \multicolumn{4}{|c|}{a semna-ez - zu signieren} \\
      \hline
      \hline
      Eu semnez & Ich signiere & Noi semnăm & Wir signieren\\
      \hline
      Tu semnezi & Du signierst & Voi semnați & Ihr signiert\\
      \hline
      El semnează & Er signiert & Ei semnează & Sie signieren\\ 
      Ea semnează & Sie signiert & Ele semnează & Sie signieren\\
      \hline
      \multicolumn{4}{|c|}{participiu: semnat} \\
      \hline
     \end{tabular}
\end{center}
%
\index{a ajunge}
\begin{center}
  \begin{tabular}{ |p{3.25cm}|p{3.25cm}||p{3.25cm}|p{3.25cm}| }
      \hline
      \multicolumn{4}{|c|}{a ajunge - erreichen, ankommen} \\
      \hline
      \hline
      Eu ajung & Ich erreiche & Noi ajungem & Wir erreichen\\
      \hline
      Tu ajungi & Du erreichst & Voi ajungeți & Ihr erreicht\\
      \hline
      El ajunge & Er erreicht & Ei ajung & Sie erreichen\\ 
      Ea ajunge & Sie erreicht & Ele ajung & Sie erreichen\\
      \hline
      \multicolumn{4}{|c|}{participiu: ajuns} \\
      \hline
     \end{tabular}
\end{center}
%
\index{a arăta}
\begin{center}
  \begin{tabular}{ |p{3.25cm}|p{3.25cm}||p{3.25cm}|p{3.25cm}| }
      \hline
      \multicolumn{4}{|c|}{a arăta - zeigen} \\
      \hline
      \hline
      Eu arăt & Ich zeige & Noi arătăm & Wir zeigen\\
      \hline
      Tu arăți & Du zeigst & Voi arătați & Ihr zeigt\\
      \hline
      El arătă & Er zeigt & Ei arătă & Sie zeigen\\ 
      Ea arătă & Sie zeigt & Ele arătă & Sie zeigen\\
      \hline
      \multicolumn{4}{|c|}{participiu: arătat} \\
      \hline
     \end{tabular}
\end{center}
%
\index{a găti-esc}
\begin{center}
  \begin{tabular}{ |p{3.25cm}|p{3.25cm}||p{3.25cm}|p{3.25cm}| }
      \hline
      \multicolumn{4}{|c|}{a găti-esc - kochen} \\
      \hline
      \hline
      Eu gătesc & Ich koche & Noi gătim & Wir kochen\\
      \hline
      Tu gătești & Du kochst & Voi gătiți & Ihr kocht\\
      \hline
      El gătește & Er kocht & Ei gătesc & Sie kochen\\ 
      Ea gătește & Sie kocht & Ele gătesc & Sie kochen\\
      \hline
      \multicolumn{4}{|c|}{participiu: gătit} \\
      \hline
     \end{tabular}
\end{center}
%
\index{a crede}
\begin{center}
  \begin{tabular}{ |p{3.25cm}|p{3.25cm}||p{3.25cm}|p{3.25cm}| }
      \hline
      \multicolumn{4}{|c|}{a crede - glauben} \\
      \hline
      \hline
      Eu cred & Ich glaube & Noi credem & Wir glauben\\
      \hline
      Tu crezi & Du glaubst & Voi credeți & Ihr glaubt\\
      \hline
      El crede & Er glaubt & Ei cred & Sie glauben\\ 
      Ea crede & Sie glaubt & Ele cred & Sie glauben\\
      \hline
      \multicolumn{4}{|c|}{participiu: crezut} \\
      \hline
     \end{tabular}
\end{center}
%
\index{a locui-esc}
\begin{center}
  \begin{tabular}{ |p{3.25cm}|p{3.25cm}||p{3.25cm}|p{3.25cm}| }
      \hline
      \multicolumn{4}{|c|}{a locui-esc - wohnen} \\
      \hline
      \hline
      Eu locuiesc & Ich wohnen & Noi locuim & Wir wohnen\\
      \hline
      Tu locuiești & Du wohnst & Voi locuiți & Ihr wohnt\\
      \hline
      El locuiește & Er wohnt & Ei locuiesc & Sie wohnen\\ 
      Ea locuiește & Sie wohnt & Ele locuiesc & Sie wohnen\\
      \hline
      \multicolumn{4}{|c|}{participiu: locuit} \\
      \hline
     \end{tabular}
\end{center}
%
\index{a învăța}
\begin{center}
  \begin{tabular}{ |p{3.25cm}|p{3.25cm}||p{3.25cm}|p{3.25cm}| }
      \hline
      \multicolumn{4}{|c|}{a învăța - lernen} \\
      \hline
      \hline
      Eu învăț & Ich lerne & Noi învățăm & Wir lernen\\
      \hline
      Tu înveți & Du lernst & Voi învățați & Ihr lernt\\
      \hline
      El învață & Er lernt & Ei învață & Sie lernen\\ 
      Ea învață & Sie lernt & Ele învață & Sie lernen\\
      \hline
      \multicolumn{4}{|c|}{participiu: învățat} \\
      \hline
     \end{tabular}
\end{center}
%
\index{a aștepta}
\begin{center}
  \begin{tabular}{ |p{3.25cm}|p{3.25cm}||p{3.25cm}|p{3.25cm}| }
      \hline
      \multicolumn{4}{|c|}{a aștepta - warten} \\
      \hline
      \hline
      Eu aștept & Ich warte & Noi așteptăm & Wir warten\\
      \hline
      Tu aștepți & Du wartest & Voi așteptați & Ihr wartet\\
      \hline
      El așteaptă & Er wartet & Ei așteaptă & Sie warten\\ 
      Ea așteaptă & Sie wartet & Ele așteaptă & Sie warten\\
      \hline
      \multicolumn{4}{|c|}{participiu: așteptat} \\
      \hline
     \end{tabular}
\end{center}
%
\index{a completa-ez}
\begin{center}
  \begin{tabular}{ |p{3.25cm}|p{3.25cm}||p{3.25cm}|p{3.25cm}| }
      \hline
      \multicolumn{4}{|c|}{a completa-ez - ausfüllen} \\
      \hline
      \hline
      Eu completez & Ich lerne & Noi completăm & Wir lernen\\
      \hline
      Tu completezi & Du lernst & Voi completați & Ihr lernt\\
      \hline
      El completează & Er lernt & Ei completează & Sie lernen\\ 
      Ea completează & Sie lernt & Ele completează & Sie lernen\\
      \hline
      \multicolumn{4}{|c|}{participiu: completat} \\
      \hline
     \end{tabular}
\end{center}
%
%
\index{a pregăti-esc}
\begin{center}
  \begin{tabular}{ |p{3.25cm}|p{3.25cm}||p{3.25cm}|p{3.25cm}| }
      \hline
      \multicolumn{4}{|c|}{a pregăti-esc - vorbereiten} \\
      \hline
      \hline
      Eu pregătesc & Ich bereite vor & Noi pregătim & Wir bereiten vor\\
      \hline
      Tu pregătești & Du bereitest vor & Voi pregătiți & Ihr bereitet vor\\
      \hline
      El pregătește & Er bereitet vor  & Ei pregătesc & Sie bereiten vor\\ 
      Ea pregătește & Sie bereitet vor & Ele pregătesc & Sie bereiten vor\\
      \hline
      \multicolumn{4}{|c|}{participiu: pregătit} \\
      \hline
     \end{tabular}
\end{center}
%
%
\index{a primi-esc}
\begin{center}
  \begin{tabular}{ |p{3.25cm}|p{3.25cm}||p{3.25cm}|p{3.25cm}| }
      \hline
      \multicolumn{4}{|c|}{a primi-esc - bekommen, erhalten} \\
      \hline
      \hline
      Eu primesc & Ich bekomme & Noi primim & Wir bekommen\\
      \hline
      Tu primești & Du bekommst & Voi primiți & Ihr bekommt\\
      \hline
      El primește & Er bekommt & Ei primesc & Sie bekommen\\ 
      Ea primește & Sie bekommt & Ele primesc & Sie bekommen\\
      \hline
      \multicolumn{4}{|c|}{participiu: primit} \\
      \hline
     \end{tabular}
\end{center}
%
\index{a rezerva}
\begin{center}
  \begin{tabular}{ |p{3.25cm}|p{3.25cm}||p{3.25cm}|p{3.25cm}| }
      \hline
      \multicolumn{4}{|c|}{a rezerva - reservieren} \\
      \hline
      \hline
      Eu rezerv & Ich reserviere & Noi rezervăm & Wir reservieren\\
      \hline
      Tu rezervi & Du reservierst & Voi rezervați & Ihr reserviert\\
      \hline
      El rezervă & Er reserviert & Ei rezervă & Sie reservieren\\ 
      Ea rezervă & Sie reserviert & Ele rezervă & Sie reservieren\\
      \hline
      \multicolumn{4}{|c|}{participiu: rezervat} \\
      \hline
     \end{tabular}
\end{center}
%
\index{a uita}
\begin{center}
  \begin{tabular}{ |p{3.25cm}|p{3.25cm}||p{3.25cm}|p{3.25cm}| }
      \hline
      \multicolumn{4}{|c|}{a uita - vergessen} \\
      \hline
      \hline
      Eu uit & Ich vergesse & Noi uităm & Wir vergessen\\
      \hline
      Tu uiți & Du vergisst & Voi uitați & Ihr vergesst\\
      \hline
      El uită & Er vergisst & Ei uită & Sie vergessen\\ 
      Ea uită & Sie vergisst & Ele uită & Sie vergessen\\
      \hline
      \multicolumn{4}{|c|}{participiu: uitat} \\
      \hline
     \end{tabular}
\end{center}
%
\index{a se uita}
\begin{center}
  \begin{tabular}{ |p{3.25cm}|p{3.25cm}||p{3.25cm}|p{3.25cm}| }
      \hline
      \multicolumn{4}{|c|}{a se uita - ansehen, schauen} \\
      \hline
      \hline
      Eu mă uit & Ich schaue & Noi ne uităm & Wir schauen\\
      \hline
      Tu te uiți & Du schaust & Voi vă uitați & Ihr schaut\\
      \hline
      El se uită & Er schaut & Ei se uită & Sie schauen\\ 
      Ea se uită & Sie schaut & Ele se uită & Sie schauen\\
      \hline
      \multicolumn{4}{|c|}{participiu: uitat} \\
      \hline
     \end{tabular}
\end{center}
%
A juca wird verwendet wenn man ausdrückt, dass man ein gewisses Spiel spielt. Also Fußball, Schach, \dots
\index{a juca}
\begin{center}
  \begin{tabular}{ |p{3.25cm}|p{3.25cm}||p{3.25cm}|p{3.25cm}| }
      \hline
      \multicolumn{4}{|c|}{a juca - spielen} \\
      \hline
      \hline
      Eu joc & Ich spiele & Noi jucăm & Wir spielen\\
      \hline
      Tu joci & Du spielst & Voi jucați & Ihr spielt\\
      \hline
      El joacă & Er spielt & Ei joacă & Sie spielen\\ 
      Ea joacă & Sie spielt & Ele joacă & Sie spielen\\
      \hline
      \multicolumn{4}{|c|}{participiu: jucat} \\
      \hline
     \end{tabular}
\end{center}
%
A se juca wird verwendet wenn man allegemein sagt, dass jemand spielt aber nicht genau spezifiziert wird was.
\index{a se juca}
\begin{center}
  \begin{tabular}{ |p{3.25cm}|p{3.25cm}||p{3.25cm}|p{3.25cm}| }
      \hline
      \multicolumn{4}{|c|}{a se juca - spielen} \\
      \hline
      \hline
      Eu mă joc & Ich spiele & Noi ne jucăm & Wir spielen\\
      \hline
      Tu te joci & Du spielst & Voi vă jucați & Ihr spielt\\
      \hline
      El se joacă & Er spielt & Ei se joacă & Sie spielen\\ 
      Ea se joacă & Sie spielt & Ele se joacă & Sie spielen\\
      \hline
      \multicolumn{4}{|c|}{participiu: jucat} \\
      \hline
     \end{tabular}
\end{center}
% 
%
%
%
%
\pagebreak
\section{Nützliche Quellen}
Konjugation von Verben: 
\begin{itemize}
    \item \url{https://www.cactus2000.de/roman/}\\
    \item \url{https://www.conjugare.ro/}\\
\end{itemize}
Plural von Substantiven: \url{https://www.pluralul.ro/}\\
\newline
Lesematerial:
\begin{itemize}
  \item Die Märchen der Gebrüder Grimm sind (für mich) sehr schwer. Aber trotzdem eine super Übung für die Zukunft! \url{https://www.grimmstories.com/ro/grimm_basme/index}
  \item \url{https://www.povesti-pentru-copii.com/} 
\end{itemize}

%
\newpage
\printindex
%
\end{document}  